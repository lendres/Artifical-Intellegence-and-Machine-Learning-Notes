% All statistics glossary entries.

%\newglossaryentry{mdl:}
%{
%	type=model,
%    name=,
%    description={}
%}

\newglossaryentry{mdl:bagging}
{
	type=model,
    name=bagging,
    description={Constructing models in parallel.  Also called bootstrap aggregating}
}

\newglossaryentry{mdl:boosting}
{
	type=model,
    name=boosting,
    description={Constructing models in serial}
}

\newglossaryentry{mdl:bootstrapaggregating}
{
	type=model,
    name=bootstrap aggregating,
    description={See \gls{mdl:bagging}}
}

\newglossaryentry{mdl:ensemble}
{
	type=model,
    name=ensemble techniques,
    description={Using multiple models to obtain better predictive performance}
}

\newglossaryentry{mdl:unstable}
{
	type=model,
    name=unstable,
    description={Models that are very sensitive to small changes in the data (small changes in data lead to a different model)}
}

\newglossaryentry{mdl:unsupervisedlearning}
{
	type=model,
    name=unsupervised learning,
    description={Unsupervised learning uses machine learning algorithms to analyze and cluster unlabeled data sets. These algorithms discover hidden patterns in data without the need for human intervention (hence, they are �unsupervised�).  Unsupervised learning models are used for three main tasks: clustering, association and dimensionality reduction.}
}

\newglossaryentry{mdl:oddsratio}
{
	type=model,
    name=odds ratio,
    description={The probability of successful result divided by the probability of failure result}
}

\newglossaryentry{mdl:randomforrest}
{
	type=model,
    name=random forrest,
    description={Bagging that uses decision trees as the models}
}

\newglossaryentry{mdl:supervisedlearning}
{
	type=model,
    name=supervised learning,
    description={Supervised learning is when you already know the label (value) of the target variable. It is of two types: regression (for continuous variables) and classification (for categorical or discrete values)}
} 