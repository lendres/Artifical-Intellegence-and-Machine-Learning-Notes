% All statistics glossary entries.

%\newglossaryentry{stg:}
%{
%	type=stats,
%    name=,
%    description={}
%}

\newglossaryentry{stg:attribute}
{
	type=stats,
    name=attribute,
    description={One part of a data object.  A data object is composed of one or more attributes}
}

\newglossaryentry{stg:bernoullidistribution}
{
	type=stats,
    name=bernoullidistribution,
    description={A distribution that has two possible outcomes (0 or 1).  It is a single trial.}
}

\newglossaryentry{stg:binomialdistribution}
{
	type=stats,
    name=binomialdistribution,
    description={A distribution of Bernoulli Distributions.  I.e., it is a distribution of trials that have two possible outcomes (0 or 1/failure or success)}
}

\newglossaryentry{stg:centrallimittheorem}
{
	type=stats,
    name=central limit theorem,
    description={In many cases, when independent random variables are summed up, their properly normalized sum tends toward a normal distribution even if the original variables themselves are not normally distributed}
}

\newglossaryentry{stg:chebyshevrule}
{
	type=stats,
    name=Chebyshev rule,
    description={Regardless of how the data are distributed, at least $\left(1-\frac{1}{/k^2}\right)$ of values will fall within $k$ standard deviations (for $k>1$)}
}

\newglossaryentry{stg:cumlativedistributionfunction}
{
	type=stats,
    name=cumlative distribution function,
    description={The probability that $X$ will take the value less than or equal to $x$.  It can be represented mathematically as $F_x\left(x\right)=P\left(X\leq{}x\right)$}
}

\newglossaryentry{stg:dataobjects}
{
	type=stats,
    name=data objects,
    description={An entry of a data set, it can be composed of several pieces of information (see \textit{parameters}).  Also called \textit{samples, examples, instances, data points, objects, tuples}}
}

\newglossaryentry{stg:dataset}
{
	type=stats,
    name=data set,
    description={An entry of a data set, it can be composed of several pieces of information called \textit{data objects}}
}

\newglossaryentry{stg:descriptivestats}
{
	type=stats,
    name=descriptive statistics,
    description={Numbers that are used to summarize and describe data.  They do not generalize beyond the data at hand.  Contrast with \textit{inferential} statistics}
}

\newglossaryentry{stg:inferentialstatistics}
{
	type=stats,
    name=inferential statistics,
    description={The mathematical procedures whereby we convert information about the sample into intelligent guesses about the population.  Contract with \textit{descriptive} statistics}
}

\newglossaryentry{stg:interquartilerange}
{
	type=stats,
    name=interquartile range,
    description={A measure of statistical dispersion (also be called the mid-spread, middle 50 percent, or H-spread).  It is defined as the difference between the 75th and 25th percentiles of the data $\left(\quartilethree{}-\quartileone{}\right)$}
}

\newglossaryentry{stg:kerneldensityestimation}
{
	type=stats,
    name=kernel density estimation,
    description={A non-parametric way to estimate the probability density function of a random variable.}
}

\newglossaryentry{stg:levelofsignificance}
{
	type=stats,
    name=level of significance,
    description={The probability of rejecting the null hypothesis when it is true.  It is fixed before hypothesis testing.  A common value would be 5 percent}
}

\newglossaryentry{stg:mode}
{
	type=stats,
    name=mode,
    description={The value which occurs most often}
}

\newglossaryentry{stg:nullhypothesis}
{
	type=stats,
    name=null hypothesis,
    description={The hypothesis that this is no significant change in probability (i.e., the observed difference is due to chance alone)}
}

\newglossaryentry{stg:parameter}
{
	type=stats,
    name=parameter,
    description={A numerical value associated with a population. Example: The average amount of time people spend on a website}
}

\newglossaryentry{stg:percentpointfunction}
{
	type=stats,
    name=percent point function,
    description={The percentile point (e.g. 50th percentile) that has a specific probability of success.  This finds the inverse of what the cumulative distribution function finds}
}

\newglossaryentry{stg:population}
{
	type=stats,
    name=population,
    description={The universe of possible data for a specified object. Example: People who have visited or will visit a website}
}

\newglossaryentry{stg:probabilitydistributionfunction}
{
	type=stats,
    name=probability distribution function,
    description={A function whose value at any given sample in the sample space can be interpreted as providing a relative likelihood that the value of the random variable would be close to that sample.  It applies to a continuous random variable (not discrete)}
}

\newglossaryentry{stg:probabilitymassfunction}
{
	type=stats,
    name=probability mass function,
    description={A function that gives the probability that a discrete random variable is exactly equal to some value (a discrete density function).  E.g., what is the probability of $x$ successes in $n$ trials given as probability $p$ of an individual successes}
}

\newglossaryentry{stg:pvalue}
{
	type=stats,
    name=p-value,
    description={The probability a calculated test statistic has a more extreme result than a under the null hypothesis.   A very small p-value means that such an extreme observed outcome would be very unlikely under the null hypothesis.  Therefore, the null hypothesis can be rejected}
}

\newglossaryentry{stg:qualitative}
{
	type=stats,
    name=qualitative,
    description={Data are non numeric in nature and cannot be measured}
}

\newglossaryentry{stg:quantitative}
{
	type=stats,
    name=quantitative,
    description={Data are numerical in nature and can be measured and can be classified into two: discrete and continuous}
}

\newglossaryentry{stg:range}
{
	type=stats,
    name=range,
    description={A measure of dispersion that is calculated as the difference between the maximum and minimum value in the data set}
}

\newglossaryentry{stg:sample}
{
	type=stats,
    name=sample,
    description={A selection of observations from a population. Example: People (or IP addresses) who visited a website on a specific day}
}

\newglossaryentry{stg:statistic}
{
	type=stats,
    name=statistic,
    description={A numerical value associated with an observed sample. Example: The average amount of time people spent on a website on a specific day}
}

\newglossaryentry{stg:tdistribution}
{
	type=stats,
    name=t distribution,
    description={A distribution similar to the normal distribution and used in estimating the mean of a population when the sample size is small and the population's standard deviation is unknown.  It is symmetric around zero}
}

\newglossaryentry{stg:uniformdistribution}
{
	type=stats,
    name=uniform distribution,
    description={The probability is the same across a range of values (the probability of occurrence is uniformly distributed).  Does not favor a particular outcome}
}

\newglossaryentry{stg:variance}
{
	type=stats,
    name=variance,
    description={}
} 