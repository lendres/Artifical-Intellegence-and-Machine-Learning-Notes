	\chapter{Neural Network Usage and Troubleshooting Guide}
	\section{No Loss Displayed / Loss Contains NaN}

	\begin{bulletedlist}
		\item Make sure there are no \textcode{NaN}s in the source data.
		\item Make sure the processing did not add \textcode{NaN}s.  \textcode{StandardScaler} has been known to introduce them.
	\end{bulletedlist}


	\section{Hyperparameter Turning}
There are different ways of hyperparameter tuning.  See the \textcode{Data Scientist Employee Attrition Case Study.ipynb} for examples.

	\begin{bulletedlist}
		\item Using \textcode{sklearn} classes such as \textcode{RandomSearchCV} or \textcode{GridSearchCV}.
		\begin{bulletedlist}
			\item The classifier used is the \textcode{KerasClassifier}.
			\item The tuning is otherwise done in the same manner as any other hyperparameter turning using the \textcode{sklearn} library.
		\end{bulletedlist}
		\item Using the \textcode{dask\_ml} library.
		\begin{bulletedlist}
			\item The library implements a parallelized version of \textcode{GridSearchCV}.
		\end{bulletedlist}
		\item Using the \textcode{kerastuner} library.
		\begin{bulletedlist}
			\item Contains a small set of classes for tuning.  They can be inherited to create your own.
			\item Allows you to tuner on things like number of layers which is difficult in some of the other methods.
		\end{bulletedlist}
	\end{bulletedlist}