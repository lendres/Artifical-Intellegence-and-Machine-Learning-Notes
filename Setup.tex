	% This document requires the use of Lance A. Endres's LaTeX library available at the following location:
	% https://github.com/lendres/LaTeX

    % Be able to use hypertext references.
	\usepackage[colorlinks=true,
				linkcolor=blue,
				citecolor=blue,
				urlcolor=red,
				filecolor=red,
				anchorcolor=black,
				pagecolor=blue,
				linktocpage=true]{hyperref}

	\usepackage{lesubscripts}

	\usepackage{legraphicextensions}
    % Set the search path for figures.  This way we can change it if required without having to modified
    % all of the \includegraphics calls.
	\graphicspath{{Figures/}}

	\usepackage{lequestionandanswer}
%	\useqandadoublespacing{}
%	\setlength{\qaspacebetweenqanda}{6pt}

	% HYPOTHESIS TEST TABLE.
	% Table package.
	\usepackage{letable}
	\setlength{\tablerowheight}{6mm}
	\setlength{\tabledoublelinespace}{-13pt}
	\tablecolumnheaderfont{\bfseries}

	\usepackage{longtable}

	% Used to auto number table rows.
	\usepackage{array}
	\newcounter{tablerowcounter}[table]
	\renewcommand{\thetablerowcounter}{\arabic{tablerowcounter}}
	\AtBeginEnvironment{tabular}{\setcounter{tablerowcounter}{0}}

	% A new column type to apply automatic stepping
	\newcolumntype{N}{>{\tablerow\refstepcounter{tablerowcounter}\thetablerowcounter}c}

	\newcommand*{\distributiontest}[2]{#1\vspace*{8pt}\newline#2}

	\newcommand*{\arguments}{...\hspace*{0.5pt}}


	\usepackage{lelists}
	\setlength{\listtopsep}{0pt}
	
	% No space bulleted list for use in tables.
	\newenvironment{nospacebulletedlist}
	{
		\vspace*{-1.1\baselineskip}
		\begin{list}
		{$\bullet$\hfill}
		{
			\setlength{\labelwidth}{0.40cm}
			\standardlistparams{}
			\setlength{\rightmargin}{0pt}
			\setlength{\topsep}{0pt}
			\setlength{\parskip}{0pt}
		}
	}
	{
		\vspace*{-0.9\baselineskip}
		\end{list}
	}


	\usepackage{amsmath}
	\usepackage{lemath}
	\setlength{\mathwhererightmargin}{0pt}
	\setlength{\mathdefspacelength}{5.0pt}

	\usepackage{lecode}
	\setlength{\initialcodeindent}{0pt}

	\usepackage{lelistofnotations}
	\spacingbetweennotations{0.5\baselineskip}
	\notationheadinglinesize{16pt}


	% For the file extensions, use "g," "s," and "o" as the last letter and change the first 2 to create a new glossary.
	% Optional arguments are:
	%		toc: Adds the main glossary to the table of contents.  The others are added by default.
	%		nonumberlist: Prevents the page numbers from being added to each entry in the glossaries.
	\usepackage[toc=true, nonumberlist=true, acronym]{glossaries}
	\newglossary[stg]{stats}{sts}{sto}{Statistics Glossary}
	\newglossary[mdl]{model}{mts}{mto}{Model Glossary}
	\newcommand*{\addacronym}[2]{\newacronym{#1}{\MakeUppercase{#1}}{#2}}
	\makeglossaries{}

	\usepackage[bottom]{footmisc}
	\usepackage{subcaption}
	\usepackage{times}

	% EPIGRAPHS
	% Quotes at the start of chapters.
	\usepackage{epigraph}
	\renewcommand{\epigraphflush}{center}
	\renewcommand{\epigraphrule}{0pt}
	\setlength{\epigraphwidth}{\textwidth-3in}
	\newcommand*{\formattedepigraph}[2]{\epigraph{\textit{#1}}{--- #2}}

	% FONTS
	\renewcommand*{\large}{\fontsize{13pt}{16pt}\selectfont}
	\renewcommand*{\Large}{\fontsize{14pt}{19pt}\selectfont}
	\renewcommand*{\LARGE}{\fontsize{17pt}{22pt}\selectfont}

	% INCLUDE A BIBLIOGRAPHY.
	\bibliographystyle{leplain} 